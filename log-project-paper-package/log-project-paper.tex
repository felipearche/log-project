\documentclass[10pt,twocolumn]{article}
\usepackage[margin=0.75in]{geometry}
\usepackage{microtype}
\usepackage{graphicx}
\usepackage{booktabs}
\usepackage[hidelinks]{hyperref}
\usepackage{amsmath,amssymb}

\title{Streaming, Drift-Aware Log Anomaly Detection with Calibration and Provenance}
\author{Felipe Fernández-Arche Pineda}
\date{} % leave empty for no date

\begin{document}
\maketitle

% === ABSTRACT (150--200 words; ~0.2 page) ===
\begin{abstract}
We present a lightweight, streaming approach to log anomaly detection that is
robust to distribution shift via ADWIN-based drift detection and conformal
calibration. Our implementation emphasizes reproducibility (CI, hash-locked
artifacts) and clean engineering. We evaluate on [DATASETS], reporting
TPR@1\%FPR, p95/p99 latency, and throughput (events/s), and compare a
TF-IDF+IsolationForest baseline against a micro-transformer variant.
\end{abstract}

% === 1. INTRODUCTION (~0.8 page) ===
\section{Introduction}
% Problem framing: streaming log AD is hard; distribution shift, latency and throughput constraints.
% Your claim: simple, principled pipeline with calibration + drift resets performs well and is reproducible.
% Contributions (bulleted, 3--4 items).

% === 2. BACKGROUND \& RELATED WORK (0.5--0.7 page) ===
\section{Background and Related Work}
% Briefly cover TF-IDF, Isolation Forest, micro-transformers for logs, conformal prediction,
% and ADWIN drift detection. Keep focused and cite seminal work.

% === 3. METHOD (1.2--1.4 pages) ===
\section{Method}
\subsection{System Overview}
% High-level pipeline diagram (stream -> featurize -> score -> conformal threshold -> alert; ADWIN resets).
\begin{figure}[t]
  \centering
  \IfFileExists{figures/pipeline.pdf}{\includegraphics[width=\linewidth]{figures/pipeline.pdf}}{\fbox{pipeline placeholder}}
  \caption{Streaming pipeline with conformal calibration and ADWIN-based drift handling.}
  \label{fig:pipeline}
\end{figure}

\subsection{Featurization and Models}
% TF-IDF baseline; micro-transformer variant. Explain why lightweight.

\subsection{Conformal Calibration}
% Nonconformity scores, calibration set, target FPR=1\%, thresholding; online maintenance policy.

\subsection{Drift Detection with ADWIN}
% What signal you feed to ADWIN; when it triggers, how you reset/recalibrate.

\subsection{Online Scoring Loop}
% Pseudocode-level description of the streaming loop; latency budget; batching (if any).

% === 4. EXPERIMENTAL SETUP (0.4--0.6 page) ===
\section{Experimental Setup}
% Datasets (brief), hardware, implementation notes, metrics definitions:
% TPR@1\%FPR, p95/p99 latency (ms), throughput (events/s), memory footprint.
\begin{table}[t]
  \centering
  \caption{Datasets and basic stats.}
  \label{tab:datasets}
  \begin{tabular}{lrrr}
    \toprule
    Dataset & \#Lines & \#Services & Span \\
    \midrule
    ExampleA & 1.2M & 5 & 7d \\
    ExampleB & 800k & 3 & 3d \\
    \bottomrule
  \end{tabular}
\end{table}

% === 5. RESULTS (1.2--1.6 pages) ===
\section{Results}
% Main calibrated comparison (table) -- prefer "calibrated-only" snapshot table from README.
\begin{table}[t]
  \centering
  \caption{Main results (calibrated). Best in bold.}
  \label{tab:main}
  \begin{tabular}{lrrr}
    \toprule
    Model & TPR@1\%FPR (\%) & p95 (ms) & eps \\
    \midrule
    TF-IDF + IF &  82.3 &  6.4 & 12.1 \\
    Micro-Transformer & \textbf{88.7} &  8.1 & 10.5 \\
    \bottomrule
  \end{tabular}
\end{table}

% Optionally add small plots: ROC snippet, reliability curve, latency histogram.
\begin{figure}[t]
  \centering
  \IfFileExists{figures/reliability.pdf}{\includegraphics[width=\linewidth]{figures/reliability.pdf}}{\fbox{reliability placeholder}}
  \caption{Calibration reliability (expected vs.\ observed).}
  \label{fig:reliability}
\end{figure}

% === 6. ABLATIONS \& ANALYSIS (0.3--0.5 page) ===
\section{Ablations and Analysis}
% Effect of calibration, effect of drift resets, window sizes, threshold sensitivity.

% === 7. DISCUSSION \& LIMITATIONS (0.3--0.5 page) ===
\section{Discussion and Limitations}
% Where it shines, where it doesn't (e.g., rare semantic shifts, cold-start).
% Engineering trade-offs; future work.

% === 8. REPRODUCIBILITY NOTES (0.2--0.3 page) ===
\section{Reproducibility Notes}
% Point to GitHub repo, release, CI; data/HASHES.txt; protected JSON policy; scripts to regenerate figures.
Project page and code: \url{https://github.com/felipearche/log-project}.

% === 9. CONCLUSION (0.2--0.3 page) ===
\section{Conclusion}
% One short paragraph summarizing findings and impact.

\paragraph{Acknowledgments}
% (Optional) leave empty or thank collaborators.

\bibliographystyle{ieeetr}
\bibliography{refs}
\end{document}
