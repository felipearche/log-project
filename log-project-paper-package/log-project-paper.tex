
\documentclass[11pt,twocolumn]{article}

\usepackage{times}

\usepackage{balance}
\usepackage[margin=1in]{geometry}
\usepackage{microtype}
\usepackage{booktabs}
\usepackage{graphicx}
\usepackage{hyperref}
\usepackage{cite}

\title{Streaming, Drift-Aware Log Anomaly Detection with Calibration and Provenance}
\author{Felipe Fernández-Arche Pineda}
\date{\today}

% Tighter floats for a compact 5–6 page fit
\setlength{\textfloatsep}{8pt plus 1pt minus 2pt}
\setlength{\floatsep}{8pt plus 1pt minus 2pt}
\begin{document}
\maketitle

\begin{abstract}
I built a lightweight, streaming-first log anomaly detector that stays reliable under concept drift. The system combines TF-IDF + Isolation Forest for a fast baseline with a compact transformer scorer for semantic signals. ADWIN handles drift; conformal prediction keeps the false positive rate on target. On the v0.1.2 (Paper) snapshot I see mean TPR@1%FPR ≈ 0.571 (best 1.000), p95 ≈ 3.02 ms, p99 ≈ 3.23 ms, and throughput ≈ 1175287 events/sec.
\end{abstract}

\section{Introduction}
In production, logs shift. New releases land, traffic patterns change, and what looked rare last week becomes routine.
Batch models drift out of tune and alert budgets get blown. I wanted a baseline that is fast, calibrated, and easy to reason about.

My design goals were: short feedback loops, a clean provenance trail, and guardrails around false positives.
I kept the implementation small so it fits into CI and is simple to audit later.

\section{Method}
\textbf{Ingestion and features.} I tokenize and normalize logs into a stable vocabulary, producing (i) sparse TF-IDF vectors and (ii) token sequences for a tiny transformer scorer.

\textbf{Scoring.} Baseline: TF-IDF $\rightarrow$ Isolation Forest (predictable, cold-start friendly). The transformer adds semantic sensitivity; it is lazily imported in the streaming app.

\textbf{Drift and calibration.} I monitor score distributions with ADWIN. On a drift event the app resets the model and optionally re-fits on a short buffer.
For thresholds, I use conformal quantiles to target a fixed FPR (e.g., 1\%).

\textbf{Metrics.} Primary: TPR@1\%FPR. Systems: p95/p99 latency (ms) and throughput (events/sec).

\section{Experiments}
Each run logs to \texttt{experiments/summary.csv} with a matching block in \texttt{docs/PROVENANCE.txt}.


\begin{table}
\centering
\caption{Per-group results.}
\label{tab:pergroup}
\begin{tabular}{lrrrr}
\toprule
      Group &  TPR@1\%FPR &  p95 (ms) &  p99 (ms) &        eps \\
\midrule
   baseline &        1.0 &      5.61 &       6.0 &     234.04 \\
transformer &        0.0 &      0.00 &       0.0 & 2546182.17 \\
\bottomrule
\end{tabular}
\end{table}


\begin{figure}[h]
\centering
\includegraphics[width=.8\linewidth]{tpr_by_model.png}
\caption{Mean TPR@1\%FPR by group.}
\label{fig:tpr}
\end{figure}


\section{Reproducibility}
Release: v0.1.2 (Paper). Encoding/EOL: UTF-8 (no BOM), LF. Provenance: 1:1 summary.csv $\leftrightarrow$ PROVENANCE.txt.
CI: pre-commit, mypy, CSV/provenance checker, pytest (Ubuntu + Windows, Python 3.11). Packaging: clean \texttt{git archive}; binary assets marked.

\section{Limitations}
Conformal guarantees assume exchangeability; severe non-stationarity still hurts.
The transformer costs more CPU, so I keep it small and watch p95/p99 latency.

\section{Conclusion}
Small, fast, and auditable beats clever but fragile. The release gives a clean baseline to build on; next steps are richer tokenization,
adaptive windows, and transformer fine-tuning if latency budgets allow.

\balance
\bibliographystyle{abbrv}
\bibliography{refs}
\end{document}
